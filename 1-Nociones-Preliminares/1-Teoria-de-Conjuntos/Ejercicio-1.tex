\documentclass[10pt,a4paper]{jhwhw}
\usepackage[utf8]{inputenc}
%Paquetes Necesarios
\usepackage{amsmath}
\usepackage{amsfonts}
\usepackage{amssymb}
\usepackage{makeidx}
\usepackage[spanish,es-lcroman]{babel}
\usepackage{titling}
\usepackage{amsthm}
\usepackage{enumerate}
\usepackage{tikz}
\usepackage{latexsym}
\usepackage{cite}
\usepackage{titlesec}
\usepackage{fancybox}
\usepackage{xparse}
%Quitar el identado de todos los parrafos
\setlength{\parindent}{0cm}
%Para agregar el identado en cada item de enumerate o cualquier otro, usar [\hspace{1cm}(a)]

%Comandos de Letras
\newcommand{\R}{\mathbb{R}}
\newcommand{\N}{\mathbb{N}}
\newcommand{\Z}{\mathbb{Z}}
\newcommand{\Q}{\mathbb{Q}}
\newcommand{\C}{\mathbb{C}}

%Informacion del del autor del libro y localizacion
\author{Autor: \href{https://www.facebook.com/ruller}{Raúl García}\\Pagina Web: \href{https://rull3r.github.io/}{MateTips}\\Correo: rull3r@hotmail.com}
\date{Venezuela\\ \today \\}
\title{Solucionario \\\href{https://books.google.co.ve/books?id=N0DrAAAACAAJ}{Álgebra Moderna - I.N. Herstein}\\}
%Para el indice alfabetico
\makeindex

%Marca de agua en el documento
\usepackage{draftwatermark}
\SetWatermarkText{\textsc{\href{https://rull3r.github.io/}{Visitame en MateTips}}} % por defecto Draft 
\SetWatermarkScale{1} % para que cubra toda la página
%\SetWatermarkColor[rgb]{1,0,0} % por defecto gris claro
\SetWatermarkAngle{55} % respecto a la horizontal

\begin{document}
	
\problema{ }\label{pro:1}
	\begin{enumerate}[\hspace{1cm}(a)]
		\item  Si $A$ es un subconjunto de $B$ y $B$ es un subconjunto de $C$, pruébese que $A$ es un subconjunto de $C$.
		\item  Si $B \subset A$ pruébese que $A\cup B=A$ y recíprocamente.
		\item  Si $B \subset A$ pruébese que para cualquier conjunto $C$ se tiene $B \cup C \subset A \cup C$ y $B \cap A \subset A \cap B$.
	\end{enumerate}
	\solution 
	\part
	Sea $x\in A$, como $A \subset B$ entonces $x\in B$, a su vez, $B \subset C$ luego $x\in C$ \QEPD
	\part
	Sea $x \in A \cup B$ entonces $x\in A$ o $x\in B$, como $B \subset A$ luego $x \in A$, por otro lado, tenemos $x \in A$ pero $B \subset A$ entonces $x \in A \cup B$ \QEPD 
	\part
	Sea $x \in B \cup C$ entonces $x\in B$ o $x\in C$, pero $B \subset A$ por lo tanto $x \in A$ o $x\in C$ $\Rightarrow$ $x\in A \cup C$  , por otro lado tenemos que $x \in B \cap C$ entonces $x\in B$ y $x\in C$, con $B \subset A$ concluimos que $x\in A$ y $x\in C$ $\Rightarrow$ $x\in A \cap C$ \QEPD
\end{document}